\documentclass[11pt,a4paper]{paper}
\usepackage{verbatimstyle,color,textract,fancyvrb-ex}
\RequirePackage[mathletters]{ucs}
\RequirePackage[utf8x]{inputenc}

\parindent=0pt
\parskip=\medskipamount


\IfFileExists{DejaVuSansMono.sty}
{\RequirePackage{DejaVuSansMono}}
{\typeout{Install DejaVu Fonts, please}}

\DefineVerbatimEnvironment{ml}{Verbatim}{fontsize=\scriptsize,frame=single,formatcom=\color{blue},numbers=left}

%%%

\begin{document}
This is an example of a picoml program embedded in a latex file. It uses the standard
picoml library, including the IO-monad-like type ACT.
\begin{ml}
   import "./lib/lib";
   
   let putStrLn: [Char] -> ACT ();
       putStrLn s = putStr s >> putStr "\n";
       
       promptLine: [Char] -> ACT [Char];
       promptLine prompt = putStr prompt >> getLine
   end
       
   let greeting   = "Hello. What's your name? ";
       prompt     = promptLine greeting;
       rec interact name = 
           prompt >>> 
           λ( "" -> return name
            | r  -> putStr "Hello " >> putStrLn r >> interact r
            ) 
   end
   
   let main = interact "I still don't know you"
       >>>       
       λ( name -> putStr "Goodbye, " >> putStr name >> putStrLn "." )
       >?> 
       λ( _ -> putStrLn "\nI can see you're not going to tell me." 
          >> 
          return () 
         );;
\end{ml}

That was an example of a picoml program embedded in a latex file. It uses the standard
picoml library, including the IO-monad-like type ACT. Here is some stuff in verbatim* 

\begin{verbatim*}
some stuff
\end{verbatim*}


\end{document}








